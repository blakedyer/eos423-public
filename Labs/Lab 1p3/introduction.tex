
\section*{Building stratigraphic records with matrix math}
In part \textbf{1.1} of this module, we learned about how in many instances, delta morphology could be shown to be controlled by bulk sediment transport. Mathematically, this claim means that \emph{diffusion} is a dominant process in building deltas, where the change of elevation over time is proportional to the second partial derivative of the topography with respect to space (the curvature). This equation is sometimes referred to as the \textit{hillslope application} of the \textbf{diffusion equation}:

\begin{equation*}
	\dfrac{\partial h}{\partial t} = K \dfrac{\partial^2 h}{\partial x^2} \label{eq:diff}
    \tag{diffusion equation}
\end{equation*}

\noindent We first looked at steady state solutions of the diffusion equation (i.e., where $\dfrac{\partial h}{\partial t} = 0$, through use of a moving reference frame). This allowed the derivation of an \emph{analytical} solution to the diffusion equation, and could be used to predict the bathymetry of deltas whose morphology was not changing with time.

\vspace{1em}

\noindent Of course, having tools to predict how delta morphology and coastline bathymetry can \emph{change} as conditions change, such as from variations in sediment type, relative sea level or sediment supply, would be very useful. In studies of the sedimentary rock record and the history it tells, these types of boundary conditions frequently change. This need was the motivation for part \textbf{1.2}, where you built a numerical model of bulk transport using an \emph{implicit} finite difference scheme called the Crank-Nicolson algorithm. \\

\vspace{1em}

\noindent This approach estimates how topography changes with time by taking the average of the central-difference estimate of the second partial derivative of the topography with respect to space at both the current time step and the future timestep:

\begin{equation*}
	{\color{red}h_{i}^{t+\Delta t}}
	-
	h_i^t
	=
	r~(h_{i-\Delta x}^{t}
	-
	2~h_{i}^{t}
	+
	h_{i+\Delta x}^{t}
	+
	{\color{red}
	h_{i-\Delta x}^{t+\Delta t}}
	-
	2~{\color{red}h_{i}^{t+\Delta t}}
	+
	{\color{red}h_{i+\Delta x}^{t+\Delta t}})
	\\
\end{equation*}


\noindent with $r$ known as the Fourier number:


\begin{equation*}
	r=K\dfrac{\Delta t}{2~\Delta x^2}
	\tag{Fourier number}
\end{equation*}



\noindent This equation forms a system of linear equations that can be rearranged and solved using linear algebra:

\begin{equation}
	\begin{bmatrix}
		2r+1 & -r   & 0    & \\
		-r   & 2r+1 & -r   & \\
		0    & -r   & 2r+1 & \\
	\end{bmatrix}
	\begin{bmatrix}
		\color{red}{h_0^{t+\Delta t} } \\
		\color{red}{h_i^{t+\Delta t}}  \\
		\color{red}{h_N^{t+\Delta t}}  \\
	\end{bmatrix}
	=
	\begin{bmatrix}
		1-2r & r    & 0    & \\
		r    & 1-2r & r    & \\
		0    & r    & 1-2r & \\
	\end{bmatrix}
	\begin{bmatrix}
		h_0 \\
		h_i \\
		h_N \\
	\end{bmatrix}
	+
	2r~
	\begin{bmatrix}
		b_0 \\
		b_i \\
		b_N \\
	\end{bmatrix}
\end{equation}

\noindent If we call the matrices on the right-hand side and left-hand side $A$ and $B$, respectively, we can rewrite the expression as:

\begin{equation*}
	A
	{\color{red}h^{t+\Delta t} }
	=
	Bh^{t}
	+
	b^{t}
\end{equation*}

\noindent where $A$ and $B$ are square matrices with whose side-length is equal to the length of $h$ and $b$ is a vector of boundary conditions. The first and last entries of $b$ define the boundary conditions of your model space - setting these will fix the fluxes into the left and right edges of the system, and thus set the topography at those edges. The matrix equation above is solved by multiplying both sides by $A^{-1}$:

\begin{equation*}
	{\color{red}h^{t+\Delta t} }
	=
	A^{-1}(Bh^{t}
	+
	b^{t})
\end{equation*}

\noindent Now you have the tools to solve the diffusion equation numerically. In each time step, the current topography ($h^t$) and boundary conditions ($b^t$) can be used to predict the topography in the next time step ($h^{t+\Delta t}$). This new topography, in turn, will be used to predict the next time step, and so on, until your model run ends. One interesting way to to use this model is to introduce a prescribed sedimentation term, represented here by the vector $s$:

\begin{equation*}
	{\color{red}h^{t+\Delta t} }
	=
	A^{-1}(Bh^{t}
	+
	b^{t}
	+
	s^{t})
\end{equation*}

\noindent The vector $s$ can be used to add topography to some part of your model space, and then allow that added topography to be modified through diffusion. For example, if you wish to simulate delta growth, a sediment flux term (which you define) could be added where topography intersects local sea level (which you also define). If this intersection occurs at $h_i$, then $s_i$ would become non-zero to reflect a growth in topography. As topography changes, so potentially can the location of sediment delivery.\\

\vspace{1em}

\noindent As the old adage goes, \textit{``All models are wrong, but some models are useful.''} Models will \emph{never} capture every single biological, chemical or physical process at play in the natural world that may be having an influence on the surface process you are studying.\footnote{Another good saying goes, \textit{``The best model of a cat is a cat.''}} Rather, models should be designed to capture some aspect of how the natural world works, and model experiments should be run to see how it responds to different parameter values or forcings. Although it is very simple, the numerical model you have developed for bulk sediment transport is rooted in some well-founded assumptions about how sediment is transported. Namely, that sediment fluxes should behave as \emph{diffusive fluxes,} acting to smooth topography over time.

\vspace{1em}

\noindent Thus, the model can be used to explore how topography changes under various boundary and initial conditions, and compared to real world data (for example, lithology logs from stratigraphic sections, or 2-D seismic data, measured across a shallow-to-deep transect of an ancient continental margin). If the model captures important aspects of the data, you can potentially make inferences about the processes recorded by your stratigraphic records (i.e., relative sea level rose, sediment flux increased, etc.). If the model fails to explain the data in some important way, this result suggests important processes are not currently captured by your model, and may spur more model development. Either outcome would be useful to science!\\

\vspace{1em}

\noindent So, let's get some practice using this model, examining the results, and thinking about what those results mean...
