In this course, we will explore how geologic and Earth surface processes, including tectonic, sea level and climate changes, are recorded and preserved in the stratigraphic record. Focus will be on modern and ancient case studies, with topics including basin analysis, cyclostratigraphy, process sedimentology and paleo-environmental reconstruction. Problem sets emphasize computational skills, including introductory time-series analysis, geospatial analysis and remote sensing.

The course will meet twice a week (M/Th) for lectures and on Fridays to work on the weekly assignment. You are encouraged to work together on all aspects of the weekly assignments, although copying or directly sharing code is not allowed. Each person must submit their own work (written answers, figures, etc.) and code used for all calculations. On Monday some weeks we will begin with a {\color{mygreen2}\textbf{flipped classroom:}} I will randomly select at least two students to informally present their progress on the current weekly assignment. These presentations will serve as a jumping off point for the whole class to discuss and work through remaining challenges with help from the instructor. Any remaining time on Monday and all of Thursday will be used for lectures on new material, guided group discussions of assigned readings, or in class exams.
